\documentclass{article} 

\usepackage{tabularx}
\usepackage{multirow}
\usepackage{lipsum}
\usepackage[table]{xcolor}

\begin{document}

\begin{table}
\centering
\begin{tabular}{|c|c|c|}
	\hline 
	\multicolumn{3}{|c|}{Multiple Column} \\
	\hline 
	A &  B & C \\
	\hline 
	\multirow{3}{*}{Multiple row} & Cell 1 & Cell 2 \\
	& Cell 3 & Cell 4 \\
	& Cell 5 & Cell 6 \\
	& Cell 7 & Cell 8 \\
	\hline
 \end{tabular}
\caption{Using multirow and multicolumn}
\end{table}

\clearpage

\begin{table}
\centering
\begin{tabularx}{0.9\textwidth} { 
		| >{\raggedright\arraybackslash}X 
		| >{\centering\arraybackslash}X 
		| >{\raggedleft\arraybackslash}X | }
	\hline
	A &  B &  C \\
	\hline 
	\lipsum[1][1] & Cell 2 & Cell 3 \\
	\hline
	Cell 4  & \lipsum[1][1]  & Cell 6  \\
	\hline
	Cell 7  & Cell 8  & \lipsum[1][1]  \\
	\hline
\end{tabularx}
\caption{A table with different alignments}
\end{table}

\clearpage

\newcolumntype{s}{>{\columncolor[HTML]{304FFE}} >{\raggedright\arraybackslash}X }
\arrayrulecolor[HTML]{00C853}
\begin{table}
  \begin{tabularx}{0.9\textwidth} { 
    | s
    | >{\raggedright\arraybackslash}X 
    | >{\raggedright\arraybackslash}X | }
  \hline
  \rowcolor{lightgray} A &  B &  C \\ \hline 
  \lipsum[1][1] & \lipsum[1][1] & \lipsum[1][1] \\ \hline
  \lipsum[1][1] & \lipsum[1][1] & \lipsum[1][1] \\ \hline
  \lipsum[1][1] & \lipsum[1][1] & \cellcolor[HTML]{F57C00} \lipsum[1][1] \\
  \hline
  \end{tabularx}
\caption{A colorized table}
\end{table}

\end{document}